\documentclass[a4paper]{article}

\usepackage[utf8]{inputenc}
\usepackage[T1]{fontenc}
\usepackage{textcomp}
\usepackage{amsmath, amssymb, amsthm}

\usepackage{setspace}
\usepackage{tikz}

\usetikzlibrary{automata, arrows, chains}
\tikzset{
	>=stealth, % makes the arrow heads bold
	node distance=3cm, % specifies the minimum distance between two nodes. Change if necessary.
	every state/.style={thick, fill=gray!10}, % sets the properties for each ’state’ node
	initial text=$ $, % sets the text that appears on the start arrow
	in place/.style={
		auto=false,
		inner sep=3pt,
	},
}

% figure support
\usepackage{import}
\usepackage{xifthen}
\pdfminorversion=7
\usepackage{pdfpages}
\usepackage{transparent}
\newcommand{\incfig}[1]{%
	\def\svgwidth{\columnwidth}
	\import{./figures/}{#1.pdf_tex}
}

\pdfsuppresswarningpagegroup=1

\title{HW: Introduction to R and Statistical Thinking}
\date{Sep 06, 2022}
\author{Jack Ruder}


\begin{document}

\doublespacing
\maketitle

\section*{1}
\begin{verbatim}
> head(trees,6)
  Girth Height Volume
1   8.3     70   10.3
2   8.6     65   10.3
3   8.8     63   10.2
4  10.5     72   16.4
5  10.7     81   18.8
6  10.8     83   19.7
\end{verbatim}

\section*{3}
The sample mean of timber yielded from the trees is approximately 30.17 cubic feet.
\subsection*{a}
The response variable is the blood pressure of the participants (systolic and diastolic). The explanatory variable is whether the participants were given white chocolate or dark chocolate.
\subsection*{b}
This study is an experiment, researchers split a group into two and controlled (to some extent) what participants ate in order to observe an effect.
\subsection*{c}
The high-blood-pressure participants are the sample, the population is people whose blood pressure is around (loosely, no data is provided here) 153/84.
\subsection*{d}
The aim of the model was to evaluate a treatment.
\subsection*{e}
Let \(S\), \(D\) be systolic and diastolic pressures respectively. Let \(C = 0\) represent eating white chocolate, and let \(C = 1\) represent eating dark chocolate.
The form of the model is thus 
\begin{align*}
	S_{end} &= S_{initial} - C \cdot (\mu_S + \epsilon) \sim f_S(\mu_S) \\
	D_{end} &= D_{initial} - C \cdot (\mu_D + \epsilon) \sim f_D(\mu_D) \\
\end{align*}
where \(f_S\) and \(f_D\) are unknown distributions of the decreases in blood pressure, distributed with mean \(\mu_i\). 
\subsection*{f}
The given data is \(S_{initial} = 153\), \(\mu_S = 5\), \(D_{initial} = 84\), \(\mu_S = 2\). Without any more datapoints it would be incorrect to pick a particular distribution of data, there is not enough information here. For this particular experimint, we have 
\begin{align*}
	S_{end} &= 153- C \cdot f_S(5), \\
	D_{end} &= 84 - C \cdot f_D(2). \\
\end{align*}
\subsection*{g}
Again, there is too little information about the statistical significance of the results. But, if the result was statistically significant then causation could be esdablished because this was a controlled experiment.
\end{document}
